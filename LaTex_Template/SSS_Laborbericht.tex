%---------------
%╔═╗╔═╗╔╦╗╦  ╦╔═╗
%╚═╗║╣    ║  ║  ║╠═╝
%╚═╝╚═╝  ╩  ╚═╝╩  
%---------------

% language setup
\newcommand{\docLanguage}{ngerman}
%\newcommand{\docLanguage}{english}

% DOCUMENT SETUP
\documentclass[12pt, oneside, a4paper, \docLanguage]{report}
\usepackage[left=3cm, 
			right=2.5cm, 
			top=2.5cm, 
			bottom=2.5cm, 
			includehead, 
			includefoot]{geometry}

% line spacing
\usepackage{setspace}
\setstretch{1,25} % 15/12 --> 1.25

% encoding setup
% T1 font encoding for languages that use a latin alphabet
\usepackage[T1]{fontenc} 

% enhanced input encoding handling - utf8 for äÄüÜöÖß...
\usepackage[utf8]{inputenc}

%de­fines Adobe Times Ro­man as de­fault text font
\usepackage{mathptmx}
\usepackage{times} % needed for acronym package

%PDF linking package
\usepackage[hidelinks]{hyperref}


% Language Setup
\usepackage[\docLanguage]{babel}
% after babel - set chapter string
\AtBeginDocument{\renewcommand{\chaptername}{}}

% language specific bibliography style
\usepackage[numbers, square]{natbib}
%\setcitestyle{square,aysep={},yysep={;}}
\usepackage[fixlanguage]{babelbib}
\selectbiblanguage{\docLanguage}
% bliographystyle setup
% babel specific: babplain, babplai3, babalpha, babunsrt, bababbrv, bababbr3
\bibliographystyle{babunsrt}


% enumeration
\usepackage{enumitem}
% tabular extension tabularx
\usepackage{tabularx}

% math packages
\usepackage{amsmath}
\usepackage{nicefrac}
\usepackage{amsthm}
\usepackage{amsbsy}
\usepackage{amssymb}
\usepackage{amsfonts}
%\usepackage{MnSymbol}


%special characters
\usepackage{amssymb}
\usepackage{upgreek,textgreek}

% acronym package
\usepackage[printonlyused, footnote]{acronym}

% breakable text in \seqsplit{}
\usepackage{seqsplit}

% \textmu
\usepackage{textcomp}

% package provides a way to compile sections of a document using the same preamble as the main document
\usepackage{subfiles}

% driver-independent color extension - used by listings,tabularx
\usepackage[usenames,dvipsnames,table,xcdraw]{xcolor}

% -- SYNTAX HIGHLIGHTING --
\usepackage{listings}
%\input{cfgs/listings/listings_def_lang_bash-cmd.tex} % adds style BASH_CMD
%\input{cfgs/listings/listings_def_lang_bash-script.tex} % adds style BASH_SCRIPT
\input{cfgs/listings/listings_def_lang_latex.tex} % adds style LATEX
%\input{cfgs/listings/listings_def_lang_matlab.tex} % adds style MATLAB
\input{cfgs/listings/listings_def_lang_python.tex} % adds style PYTHON
%\input{cfgs/listings/listings_def_lang_c++.tex} % adds style CPP
%\input{cfgs/listings/listings_def_lang_c.tex} % adds style C
%\input{cfgs/listings/listings_def_lang_json.tex} % adds style JSON

% HEADLINE CFG
\usepackage{fancyhdr} % Headers and footers
\usepackage{lastpage}
\usepackage{ifthen}
\setlength{\headheight}{1.5cm}
%\pagestyle{fancy} % All pages have headers and footers
% override plain page style for \part, \chapter or 
% \maketitle, which implicit specifies plain page style
\input{cfgs/fancyhdr/fancyhdr_pagestyle_plain.tex}
% set list pagestyle
\input{cfgs/fancyhdr/fancyhdr_pagestyle_preface.tex}
% set default pagestyle
\input{cfgs/fancyhdr/fancyhdr_pagestyle_default_onepage.tex}
%\input{cfgs/fancyhdr/fancyhdr_pagestyle_default_twopage.tex}

\renewcommand{\chaptermark}[1]{\markright{#1}{}}
\renewcommand{\sectionmark}[1]{\markright{#1}{}}
\renewcommand{\headrulewidth}{0pt}
\renewcommand{\footrulewidth}{0pt}

% PICTURE CFG 
\usepackage{verbatim}
\usepackage{graphicx}
\usepackage{epstopdf}
\usepackage{caption}
\usepackage[list=true,listformat=simple]{subcaption}
% floating prevention packages
\usepackage{float}    % used with [H] positioning parameter
\usepackage{placeins} % \FloatBarrier 
% tikz packages
\usepackage{tikz}
\usepackage{standalone}
\usepackage{pgfplots}


% include only specified tex files - uncommend here
\includeonly{preface/cover,
             preface/abstract,
             preface/tableofcontents,
             preface/listoffigures,
             preface/listoftables,
             preface/lstlistoflistings,
             appendix/bibliography}

%-------------------
%╔═╗╔╦╗╦═╗╦ ╔╗╔╔═╗╔═╗
%╚═╗  ║  ╠╦╝║ ║║║║ ╦ ╚═╗
%╚═╝  ╩  ╩╚═╩ ╝╚╝╚═╝╚═╝
%-------------------
\newcommand{\strLecture}{Signale, Systeme und Sensoren}
\newcommand{\strDate}{\today}
\newcommand{\strAuthorA}{Kiattipoom Pensuwan}
\newcommand{\strAuthorB}{Thanh Son Dang}
%\newcommand{\strAuthorC}{C. Author}
\newcommand{\strAuthorAEmail}{ki851pen@htwg-konstanz.de}
\newcommand{\strAuthorBEmail}{th851dan@htwg-konstanz.de}
%\newcommand{\strAuthorCEmail}{cauthor@htwg-konstanz.de}
% Versuchsbeschreibung 
\newcommand{\strTopic}{VERSUCH NAME}
\newcommand{\strAbstract}{Zur Überprüfung der Qualität der digitalen Kamera, nehmt man eine stufenformige Grauwertverlauf auf, die innerhalb jeder Stufe gleiche Wert haben sollte.  Mit Hilfe con Python paket OpenCV-Python kann man die Belichtungsparameter der Kamera ändern und das Bild in verlustfreien png format aufnehmen und die Information(RGB-Werte)  von Bildpunkte auslesen. Damit kann man mit Dunkelbild und Weißbild Bildfehlern und Sensorrauschen suchen.}
% hyperref customization
\hypersetup{
	pdftitle     = {\strTopic}, % title
	pdfsubject   = {\strLecture}, % subject of the document
	pdfauthor    = {\strAuthorA, \strAuthorB}, % author
	pdfkeywords  = {}, % list of keywords
	pdfcreator   = {}, % creator of the document
	pdfproducer  = {}, % producer of the document
	colorlinks   = false, % false: boxed links; true: colored links
	linkcolor    = red, % color of internal links (change box color with linkbordercolor)
    citecolor    = green, % color of links to bibliography
    filecolor    = magenta, % color of file links
    urlcolor     = cyan, % color of external links
	%bookmarks    = true, % show bookmarks bar?
	unicode	     = true, % non-Latin characters in Acrobat’s bookmarks
	pdftoolbar   = true, % show Acrobat’s toolbar?
	pdfmenubar   = true, % show Acrobat’s menu?
    pdffitwindow = false, % window fit to page when opened
	pdfnewwindow = true % links in new PDF window
}

%-----------------------------------------
% ╔╗  ╔═╗╔═╗ ╦ ╔╗╔  ╔╦╗╔═╗╔═╗╦ ╦╔╦╗╔═╗╔╗╔╔╦╗ 
% ╠╩╗║╣  ║ ╦  ║ ║║║     ║║║ ║║  ║ ║║║║║╣ ║║║ ║  
% ╚═╝╚═╝╚═╝ ╩ ╝╚╝  ═╩╝╚═╝╚═╝╚═╝╩ ╩╚═╝╝╚╝ ╩  
%-----------------------------------------

\begin{document}
\pagenumbering{Roman} 

\setcounter{section}{0}
\include{preface/cover}

\include{preface/abstract}
\clearpage

%
% TABLE OF CONTENTS
%
\include{preface/tableofcontents}

%
% Abbildungsverzeichnis
%
\include{preface/listoffigures}

%
% Tabellenverzeichnis
%
\include{preface/listoftables}

%
% Listingverzeichnis
%
\include{preface/lstlistoflistings}


%--------------------------
% ╔═╗╦  ╦╔═╗╔═╗╔╦╗╔═╗╦═╗╔═╗ 
% ║    ╠═╣╠═╣╠═╝  ║   ║╣  ╠╦╝╚═╗ 
% ╚═╝╩  ╩╩  ╩╩      ╩   ╚═╝╩╚═╚═╝ 
%--------------------------

\pagenumbering{arabic} 
\setcounter{page}{1} 
\pagestyle{default}

%
% CHAPTER Versuch 1
%
\chapter{Aufnahme und Analyse eines Grauwertkeiles}
\label{chap:VERSUCH_1}

\section{Fragestellung, Messprinzip, Aufbau, Messmittel}
\label{chap:VERSUCH_1_FRAGESTELLUNG}
eine Pyton Skript schreiben (siehe Code 1) die dass Bild mit Hilfe der OpenCV aufnehmen. 
die Position und die Distanz zwischen der Digitale Kamera (in diesem fall Webcam) und das Grauwertkeil so einzustellen, dass die möglichst komplette Grauwertverlauf in das Bild befindet. Dabei sollte die Grauwertstufen parallel zu Bildränder verlaufen.
Mit OpenCV die Belichtungparameter einstellen, dass die weiße Bereich des Bild kein Überlauft hat (maxmalwert nicht 255) da sie Information in Überlauf verloren gehen. Diese Einstellung (Distanz und Belichtungsparameter) benutzt man für alle folgenden Versuche. die aufgenemene Bild(Farbbild) in ein Grauwertbild umwandeln mit cv2.cvtColor(). 
die Grauwertverlauf in 5 Grauwertstufe teilen und als Bild speichern. Von jeder Grauwertstufe sind die Mittelwert und Standard abweichung zu ermitteln.
\begin{figure}[H]
	\centering\small
	\includegraphics[width=11cm]{versuch_aufbau.jpg}
	\caption{Versuchaufbau}
\end{figure}
\section{Messwerte}
\label{chap:VERSUCH_1_MESSWERTE}

\section{Auswertung}
\label{chap:VERSUCH_1_AUSWERTUNG}
\begin{table}[H]
\begin{tabular}{|l|l|}
\hline
\multicolumn
{1}{|c|}{framewidth}	& \multicolumn{1}{c|}{640}			\\ \hline
frameheight						&$480$					\\ \hline
brightness						&$130$					\\ \hline
contrast							&$30$					\\ \hline
saturation						&$64$					\\ \hline
gain								&$0$					\\ \hline
exposure						&$-4$					\\ \hline
white balance					&$4980$				\\ \hline
\end{tabular}
\caption{Kameraeinstellung}
\end{table}

\textit{Mittelwert: }	$\bar{x} = \frac{1}{n}\sum_{i=1}^nx_i$
\textit{Standardabweichung: }$s=\sqrt{\frac{1}{n-1}\sum_{i=1}^n{(\bar{x}-x_i)}^2}$
\begin{table}[H]
\begin{tabular}{|l|l|l|}
\hline
\multicolumn{1}{|c|}{} & \multicolumn{1}{c|}{Mittelwert} & \multicolumn{1}{c|}{Standardabweichung}\\ \hline
Grauwertstufen 1						&$212.504769$		&$5.643870$					\\ \hline
Grauwertstufen 2						&$170.716981$		&$6.199961$					\\ \hline
Grauwertstufen 3						&$128.680235$		&$4.864795$					\\ \hline
Grauwertstufen 4						&$83.420426$		&$4.632894$					\\ \hline
Grauwertstufen 5						&$41.393944$		&$2.515164$					\\ \hline
\end{tabular}
\caption{Mittelwert und Standard abweichung von Grauwertstufen (Grauwertstufen 1 die hellste bis Grauwertstufen 5 die dunkelste)}
\end{table}

\section{Interpretation}
\label{chap:VERSUCH_1_INTERPRETATION}

%
% CHAPTER Versuch 2
%
\chapter{Versuch 2}
\label{chap:VERSUCH_2}

\section{Fragestellung, Messprinzip, Aufbau, Messmittel}
\label{chap:VERSUCH_2_FRAGESTELLUNG}

\section{Messwerte}
\label{chap:VERSUCH_2_MESSWERTE}

\section{Auswertung}
\label{chap:VERSUCH_2_AUSWERTUNG}

\section{Interpretation}
\label{chap:VERSUCH_2_INTERPRETATION}

%
% CHAPTER Versuch 3
%
\chapter{Versuch 3}
\label{chap:VERSUCH_3}

\section{Fragestellung, Messprinzip, Aufbau, Messmittel}
\label{chap:VERSUCH_3_FRAGESTELLUNG}

\section{Messwerte}
\label{chap:VERSUCH_3_MESSWERTE}

\section{Auswertung}
\label{chap:VERSUCH_3_AUSWERTUNG}

\section{Interpretation}
\label{chap:VERSUCH_3_INTERPRETATION}

%
% CHAPTER Versuch 4
%
\chapter{Versuch 4}
\label{chap:VERSUCH_4}

\section{Fragestellung, Messprinzip, Aufbau, Messmittel}
\label{chap:VERSUCH_4_FRAGESTELLUNG}

\section{Messwerte}
\label{chap:VERSUCH_4_MESSWERTE}

\section{Auswertung}
\label{chap:VERSUCH_4_AUSWERTUNG}

\section{Interpretation}
\label{chap:VERSUCH_4_INTERPRETATION}
%
% CHAPTER Anhang
%
\renewcommand\thesection{A.\arabic{section}}
\renewcommand\thesubsection{\thesection.\arabic{subsection}}

\chapter*{Anhang}
\label{chap:APPENDIX}
\addcontentsline{toc}{chapter}{Anhang}
%\setcounter{chapter}{0}
\addtocounter{chapter}{1}
\setcounter{section}{0}

\section{Quellcode}
\label{chap:APPENDIX_SOURCECODE}

\subsection{Quellcode Versuch 1}
\label{chap:APPENDIX_SOURCECODE_V1}
\begin{lstlisting}[
style=PYTHON,
frame=single,
caption=Quellcode für Kameraeinstellung und Bild aufnehmen,
captionpos=b,
label=lst:V1]
import numpy as np
import cv2

cap = cv2.VideoCapture(0)
cap.set(11,30)
cap.set(10,130)
cap.set(14,0)
cap.set(15,-4)
cap.set(17,4980)

print("framewidth:" + str(cap.get(3)))
print("frameheight:" + str(cap.get(4)))
print("--------------------------------")
print("brightness:" + str(cap.get(10)))
print("contrast:" + str(cap.get(11)))
print("saturation:" + str(cap.get(12)))
print("--------------------------------")
print("gain:" + str(cap.get(14)))
print("exposure:" + str(cap.get(15)))
print("--------------------------------")
print("white_balance:" + str(cap.get(17)))


while(True):
    ret, frame = cap.read()
    gray = cv2.cvtColor(frame, cv2.COLOR_BGR2GRAY)
    cv2.imshow('frame', gray)
    if cv2.waitKey(1) & 0xFF == ord('q'):
        cv2.imwrite('bildweiß10.png',gray)
        print(np.min(gray),np.max(gray))
        break;
        
cap.release()
cv2.destroyAllWindows()
\end{lstlisting}

\subsection{Quellcode Versuch 2}
\label{chap:APPENDIX_SOURCECODE_V2}

\subsection{Quellcode Versuch 3}
\label{chap:APPENDIX_SOURCECODE_V3}

\subsection{Quellcode Versuch 4}
\label{chap:APPENDIX_SOURCECODE_V4}


\section{Messergebnisse}
\label{chap:APPENDIX_MEASUREMENT_SOURCE}

%
% Literaturverzeichnis
%
\include{appendix/bibliography}

\end{document}
%------------------------------------
% ╔═╗╔╗╔╔╦╗  ╔╦╗╔═╗╔═╗╦  ╦╔╦╗╔═╗╔╗╔╔╦╗
% ║╣  ║║║  ║║     ║║║  ║║    ║  ║║║║║ ╣  ║║║ ║ 
% ╚═╝╝╚╝═╩╝  ═╩╝╚═╝╚═╝╚═╝╩  ╩╚═╝╝╚╝  ╩ 
%------------------------------------